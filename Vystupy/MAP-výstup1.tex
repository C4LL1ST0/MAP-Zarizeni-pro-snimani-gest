\documentclass[12pt, a4paper]{article}
\usepackage[utf8]{inputenc}
\usepackage[T1]{fontenc}
\usepackage[czech]{babel}
\usepackage[hyphens]{url}
\usepackage[colorlinks=true, urlcolor=blue, linkcolor=black,
breaklinks]{hyperref}
\usepackage{tabularx}
\usepackage[table]{xcolor}
\usepackage{longtable}
\renewcommand{\arraystretch}{1.5}
\usepackage{caption}
\usepackage{graphicx}
\usepackage{float}
\graphicspath{ {.//} }

\sloppy

\begin{document}

\begin{titlepage}
\begin{center}
\thispagestyle{plain}
\pagenumbering{gobble}
\vspace*{1.0cm}

\textbf{\LARGE{1. Výstup Maturitní práce}}
\vspace{2cm}
\hrule
\vspace{0.5cm}
{\Huge \textbf{Zařízení na snímání gest}}
\vspace{0.5cm}
\hrule
\vspace{3cm}

\begin{minipage}{0.4\textwidth}
\begin{flushleft} 
  \large{\emph{autoři:}\\ Štěpán Bílek, \newline Lukáš Karásek}
\end{flushleft}
\end{minipage}
\begin{minipage}{0.4\textwidth}
\begin{flushright}
  \large{\emph{konzultant:} \\ Jaroslav Kořínek}
\end{flushright}
\end{minipage}
\vfill
\large{Školní rok 2024/2025}
\end{center}
\end{titlepage}


\pagenumbering{Roman}
\setcounter{page}{2}

\section{Úvod}
\paragraph{}
Náš první výstup maturitní práce obsahuje hardware a zpracování dat. Konkrétněji 3D model krabičky ve které vše bude, výběr a nákup komponent a nakonec návrh a objednání PCB.

\section{Hardware}
\subsection{3D modely} \label{3D modely}
\paragraph{}
Až na elektrické části bude celé zařízení tisknuté na 3D tiskárně, z čehož vyplývá, že si modely zařízení musíme vymodelovat sami, a to kvůli námi navrženému PCB, konkrétně kvůli jeho rozměrům. Na zařízení jsou potřeba 2 modely, a to tělo a víko, s tím se pojí vymyšlení a implementace způsobu jak tyto části spojit tak, aby byli oddělitelné ale zároveň drželi dostatečně pevně při sobě. Toho jsme docílili pomocí systému s drobným zpětným háčkem na těle a odpovídajícím protikusem na víku.
\begin{figure}[H]
  \centering
  \includegraphics[width=\textwidth]{MAP-krabicka-v2.png}
  \caption{Tělo krabičky-model}
\end{figure}

\begin{figure}[H]
  \centering
  \includegraphics[width=\textwidth]{MAP-vycko-v1.png}
  \caption{Víko krabičky-model}
\end{figure}





\subsection{Komponenty} \label{Komponenty}
\paragraph{}
Komponenty jsme vybrali následující:
\begin{enumerate}
    \item \textbf{ESP32 C3} - mikrokontroler
    \item \textbf{MPU-6050} - gyroskop
    \item \textbf{KLS7-SS02} - posuvný spínač
    \item \textbf{Mouser Electronics tlačítka}
    \item \textbf{LED Diody}
\end{enumerate}


\subsection{PCB} \label{PCB}
\paragraph{}
Tištěný spoj byl navržen dle zásady: "V jednoduchosti je krása." Jeho úkolem je propojit nejen elektrické součástky, ale také už hotové komponenty(esp32, mpu6050). Bylo potřeba nakresli vlastní obtisky součástek i schematické symboly, aby bylo možno udělat v softwaru KiCad kompletní návrh. Tištěný spoj je dvouvrstvý a základním materiálem je standardní dielektrikum FR-4. Spoj byl vyroben firmou JLCPCB.

\begin{figure}[H]
  \centering
  \includegraphics[width=\textwidth]{PCB.png}
  \caption{PCB-návrh}
\end{figure}

\begin{figure}[H]
  \centering
  \includegraphics[width=\textwidth]{1000005469.jpg}
  \caption{PCB}
\end{figure}

\begin{figure}[H]
  \centering
  \includegraphics[width=\textwidth]{1000005474.jpg}
  \caption{PCB v krabičce}
\end{figure}


\section{Zpracování dat} \label{Zprac}
\paragraph{}
Zařízení bude fungovat následovně mikrokontroler přečte data čipu MPU6050 a ta následně odešle do počítače uživatele. MPU6050 poskytuje data z gyroskopu a akcelerometru pro každou ze tří prostorových os. Celkově se tedy jedná o 6 hodnot, které nás zajímají. Tyto hodnoty budou odesílány jako jednoduché uspořádané pole, jelikož se jedná o nejjednodušší a nejefektivnější řešení. Vyčítání a uspořádání jsme odzkoušeli.

\newpage
\section{Dělba práce}
\paragraph{}
Odvedená práce byla rozdělena následujícím způsobem:

\textbf{Lukáš}
\begin{itemize}
    \item \ref{3D modely} \textit{3D modely}
    \item \ref{Komponenty} \textit{Komponenty}
\end{itemize}

\textbf{Štěpán}
\begin{itemize}
    \item \ref{PCB} \textit{PCB}
    \item \ref{Zprac} \textit{Zpracování dat}
\end{itemize}
\vspace{5cm}
\tableofcontents
\end{document}
