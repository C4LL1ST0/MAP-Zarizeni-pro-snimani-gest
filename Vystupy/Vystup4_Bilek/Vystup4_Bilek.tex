\documentclass[12pt, a4paper]{article}
\usepackage[a4paper, margin=2.5cm]{geometry}
\usepackage{setspace}
\setlength{\parindent}{1.25cm}
\setlength{\parskip}{0pt}
\onehalfspacing
\usepackage[utf8]{inputenc}
\usepackage[czech]{babel}
\usepackage[hyphens]{url}
\usepackage[colorlinks=true, urlcolor=blue, linkcolor=black, breaklinks, citecolor=black]{hyperref}
\usepackage{caption}
\usepackage{graphicx}
\usepackage{float}
\graphicspath{ {.//obrazky/} }
\usepackage[backend=biber]{biblatex}
\addbibresource{citace.bib}


\sloppy

\begin{document}

\begin{titlepage}
\begin{center}
\thispagestyle{plain}
\pagenumbering{gobble}
\vspace*{1.0cm}

\textbf{\LARGE{4. Výstup Maturitní práce}}
\vspace{2cm}
\hrule
\vspace{0.5cm}
{\Huge \textbf{Zařízení na snímání gest}}
\vspace{0.5cm}
\hrule
\vspace{3cm}

\begin{minipage}{0.4\textwidth}
\begin{flushleft}
  \large{\emph{autor:}\\ Štěpán Bílek}
\end{flushleft}
\end{minipage}
\begin{minipage}{0.4\textwidth}
\begin{flushright}
  \large{\emph{konzultant:} \\ Jaroslav Kořínek}
\end{flushright}
\end{minipage}
\vfill
\large{Školní rok 2025/2026}
\end{center}
\end{titlepage}

\pagenumbering{Roman}
\setcounter{page}{2}



\section{Úvod}
\paragraph{}
Cílem 4. výstupu maturitní práce bylo zkompletovat a otestovat vše tzn. senzorické zařízení, firmware, ovládací program jakožto celek.
Zárověn bylo nutno vyřešit napájení z baterie, nýbž dříve navržené řešení nebylo funkční.

\section{Napájení}
\paragraph{}
Původním plánem bylo napájet zařízení ze 3 knoflíkových bateríí LR44 řazených v sérii, jelikož senzorické zařízení potřebuje 3,3V - 6,3V.
Toto řešení jsme chtěli vyzkoušet kvůli tomu, že tyto baterie jsou velmi malé a jednoduché na použití, také by při vybití šly snadno vyměnit.
Při testování jsem ale zjistil, že nedokáží poskytovat dostatečný proud pro napájení a tudíž jsou nevhodné.

Se spolupracovníkem LK jsme vybrali a koupili nový Li-po článek, který jsem k zařízení připájel. Zároveň jsme museli koupit
i nabíječku.
\begin{figure}[H]
  \centering
  \includegraphics[width=7cm]{zarizeni.jpg}
  \caption{zařízení s akumulátorem}
\end{figure}

\begin{figure}[H]
  \centering
  \includegraphics[width=7cm]{nabijecka.jpg}
  \caption{nabíječka}
\end{figure}

Nyní je zařízení napájené z baterie plně funkční, pouze na nabíječku bude potřeba vytisknout krabičku, jejíž model jsem už udělal.

\begin{figure}[H]
  \centering
  \includegraphics[width=7cm]{krabicka.png}
  \caption{model pouzdra nabíječky}
\end{figure}


\section{Testování}
\paragraph{}
Při prvním testování vyhodnocení gest modelem bylo zjištěno, že model vše vyhodnotí jako \uv{doprava}, po
přůzkumu s pomocí ChatGPT\cite{ChatGPT} jsem zjistil, kde by mohl být problém. Po přečtení dokumentace \cite{tentorflow}
jsem ho opravil. Naštěstí se jednalo pouze o špatné použití funkce knihovny pro rozřazení trénovacích dat.

Když jsem ale rozpoznávání gest zkoušel i po té, nebylo zrovna spolehlivé, alespoň ne na mně. Jelikož všechna tránovací data
doposud nahrával LK, rozpoznávání u něj fungovalo bezchybně. Musel jsem tedy rozšířit trénovací data o gesta i já. Dohromady
náš dataset nyní čítá 600 gest od každého 300, vždy 150 \uv{doleva} a 150 \uv{doprava}.

\section{Optimalizace firmwaru}
\paragraph{}
Aby model UI mohl gesto správně vyhodnotit potřebuje k tomu dostatek dat ze senzoru. Délku gesta jsme stanovili na 1s a
zařízení se snažilo data odesílat 50x. V průměru ovládací program obdrzěl ale pouze 38 \uv{snímků}/objektů za sekundu, čili jedno gesto.
Po optimalizaci firmwaru, která zahrnovala odmazání přebytečných proměnných a odstranění testovací serial komunikace, zařízení
konzistentně odesílá 48 objektů/s. Je důležité zmínit, že s délkou gesta se mění počet vstupů neuronové sítě; nyní 1s * 48objektů
* 6 hodnot(v každém objektu).

Aby se všechna trénovací data nemusela nahrávat znovu, udělal jsem program, který předešlá data zkrátí.


\vspace{1cm}
\noindent
\textbf{poznámka: Veškeré soubory k projektu jsou veřejně k dispozici na platformě github \cite{github}.}


\newpage
\printbibliography
\tableofcontents

\end{document}
