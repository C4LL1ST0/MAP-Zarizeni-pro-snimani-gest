\documentclass[11pt, a4paper]{article}
\usepackage[utf8]{inputenc}

\usepackage[czech]{babel}
\usepackage[hyphens]{url}
\usepackage[colorlinks=true, urlcolor=blue, linkcolor=black, breaklinks, citecolor=black]{hyperref}
\usepackage{graphicx}
\usepackage{float}

\graphicspath{ {.//obrazky/} }
\usepackage[backend=biber]{biblatex}
\addbibresource{citace.bib}



\sloppy

\begin{document}

\begin{titlepage}
  \begin{center}
    \thispagestyle{plain}
    \pagenumbering{gobble}
    \vspace*{1.0cm}

    \textbf{\LARGE{Výstup 2 maturitní práce}}
    \vspace{2cm}
    \hrule
    \vspace{0.5cm}
    {\Huge \textbf{Zařízení na snímání gest}}
    \vspace{0.5cm}
    \hrule
    \vspace{3cm}

    \begin{minipage}{0.4\textwidth}
      \begin{flushleft}
        \large{\emph{autor:}\\ Štěpán Bílek}
      \end{flushleft}
    \end{minipage}
    \begin{minipage}{0.4\textwidth}
      \begin{flushright}
        \large{\emph{konzultant:} \\ Jaroslav Kořínek}
      \end{flushright}
    \end{minipage}
    \vfill
    \large{Školní rok 2025/2026}
  \end{center}
\end{titlepage}


\pagenumbering{Roman}
\setcounter{page}{2}

\section{Úvod}
\paragraph{}
Primárním cílem výstupu 2 maturitní práce bylo seznámit se s principem neuronových sítí a navrhnout jednu pro
ovládací program za účelem rozpoznání jednotlivých gest. Cílem sekundárním bylo začít na ovládacím programu
samotném, aby byl schopen přijímat data ze zařízení a ukládat je jako trénovací soubory pro neuronovou síť.

\section{Neuronová síť}
\subsection{Poznatky}
\paragraph{}
Umělá neuronová síť je struktura určená pro distribuované paralelní zpracování dat. Skládá se z umělých neuronů,
jejichž volným předobrazem je biologický neuron. Neurony jsou vzájemně propojeny synaptickými vazbami a navzájem
si předávají signály a transformují je pomocí aktivačních funkcí. \cite{prehled-NN}

Neurony jsou uspořádávány do vrstev. Existuje vrstva vstupní, skrytá (kterých může být více) a výstupní.

\subsubsection{Neurony, synapse a vrstvy}
\paragraph{}
Neuron je funkcí, která má libovolný počet vstupů. Tato funkce je ve své podstatě jednoduchá, přijímané vstupy se
nejdříve vynásobí vahou vázanou k synapsy, ze které přišly, dále se tyto hodnoty sečtou a nakonec je přičten tzv. \uv{bias}.

Synapse je propojení mezi neurony. Ke každé jednotlivé synapsy, kterou neuron má, je přiřazena váha, která začíná
na náhodném čísle a v procesu učení je upřesnǒvána.

\begin{figure}[h]
  \centering
  \includegraphics[width=12cm]{neuron.png}
  \caption{neuron}
\end{figure}

Za předpokladu neuronu s 1 vstupem[x], vahou na synapsy[w], bias[b] a jedním výstupem[y] vypadá jeho oprace po matematické stránce takto:
\begin{center}
  \begin{equation}
    y = x*w + b
  \end{equation}
\end{center}
\textit{pozn. Jinými slovy se jedná o přímku / lineární funkci.}
\newline
\newline
\paragraph{}
Počet vrstev a neuronů v každé z nich se stanovuje podle komplexity problému, jež má síť za úkol řešit, a velikosti
datasetu. Neexistuje obecné pravidlo pro stanovení ideálního počtu vrtev ani neuronů. Důležité je při trénování
modelu sledovat přesnost a podle toho síť adaptovat. Síť musí být dostatečně složitá, aby dokázala
produkovat správné výsledky, avšak ne tolik složitá aby si data pouze pamatovala.


\subsubsection{Aktivační funkce}
\paragraph{}
Je další funkcí v pořadí za neuronem. Tato funkce není nezbytná pro neuronovou síť, ale zpravidla dělá
výstup nelineárním, díky čemuž se síť může naučit komplexnější vzory. Nejčastěji používanými jsou: \textbf{sigmoid},
\textbf{tanh}, \textbf{ReLU} a tomu podobné, \textbf{softmax}. Pro určíté úkoly na různých vrstvách sítě
jsou některé funkce lepší, než jiné. Odvíjí se to od faktorů jako příkrost vzrůstu a H\textsubscript{f}.\cite{af}


\subsubsection{Ztrátová funkce}
\paragraph{}
Ztrátová funkce určuje, jak správně model předpovídá/určuje v porovnání s opravdovým výsledkem.
Čím menší číslo je výstup této funkce, tím přesnější model je. Tato funkce je důležitá
při učení, kdy právě podle chybovosti algoritmy upravují parametry modelu(váhy, bias).
Těchto funcí existuje mnoho a každá je lepší na něco jiného\cite{loss}.

\subsubsection{Optimalizační funkce}
\paragraph{}
Optimalizační funkce úzce souvisí se ztrátovou funkcí a na základě jejího výstupu
upravuje parametry modelu, jak již bylo výše zmíněno\cite{optim}.

\subsubsection{Implementace perceptronu}
\paragraph{}
Pro lepší pochopení celého tématu jsem implementoval jeden z nejjednoduších modelů neuronové sítě - Perceptron\cite{percp}.
\begin{figure}[H]
  \centering
  \includegraphics[width=\textwidth]{per_jup_2.png}
  \caption{implementace perceptronu}
\end{figure}

\subsection{Návrh}
\subsubsection{Vrstvy a neurony}
\paragraph{}
Neuronová síť bude sestávat ze 4 vrstev, čili bude obsahovat 2 skryté vrstvy.

Vstupní vrstva bude mít 180 vstupů, jelikož data z gyroskopu a akcelerometru jsou složena
z 6 hodnot, délku gesta budu předběžne počítat 2s a frekvenci sběru 15Hz.

První skrytá vrstva bude LSTM(Long Short-Term Memory)\cite{LSTM}, která se používá především pro
zpracování sekvenčních dat - zachcené hodnoty z gyroskopu a akcelerometru za určitý čas.
Měla by dokázat dobře zachtit a rozpoznat průběh gest. Počet neuronů jsem stanovil na 32.

Druhá je vrsva tzv. \uv{hustá}, její charakteristikou je, že všechny neurony jsou připojeny
ke všem neuronům z vrtvy předchozí\cite{dense}. Počet neuronů bude 16.

Poslední vrstva bude také hustá a bude mít 2 neuron, jelikož nás budou zajímat 2 gesta (doleva, doprava).
\newline

\textit{pozn. Počet neuronů v jednotlivých vrstvách se pravděpodobně bude v budoucnu během testování sítě měnit,
  uvedená čísla jsou počáteční.}


\subsubsection{Aktivační funkce}
\paragraph{}
U první vrstvy bude použit hyperbolický tangens, protože ho LSTM vrstva již vnitřně používá
a s jinou aktivační funkcí by vrstva nemusela produkovat správné výsledky.

U druhé skryté vrstevy bude použita funkce \textbf{leaky ReLU}, jelikož je jednoduchá na výpočet -
což bude hrát roli především u tréninku -
a oproti standartní ReLU nemůže dojít k \uv{zaseknutí} výstupu neuronu na 0.

Na vrstvě výstupní bude použita funkce softmax, která je běžně využívána jedná-li se
o klasifikaci výstupu do více tříd. Funkce každé třídě přiřadí pravděpodobnost.


\subsubsection{Ztrátová funkce}
\paragraph{}
Ztrátová funce bude \uv{Categorical Cross-Entropy Loss}, protože slouží pro modely,
které klasifikují více tříd, jejichž výstupem je pravděpodobnost.

\subsubsection{Optimalizační funkce}
\paragraph{}
Jako optimalizátor bude použit alogoritmus \uv{Adam}, především kvůli jeho univezálnosti,
dále také nepotřebuje podrobně ladit rychlost učení. Pro síť a dataset této velikosti
by měl být ideální.

\newpage
\section{Ovládací program}
\paragraph{}
Ovládací program pro zařízení na snímání gest má za úkol přjímat a vyhodnocovat data ze
zařízení na počítači uživatele a dle toho počítač ovládat(posouvat snímky prezentace).

Program je psaný v jazyce Python, kvůli jeho rozsáhlému ekosystému knihoven. Zároveň
je Python interpretovaným jazykem a tudíž ho lze spustit na jakémkoli operačním systému,
kde běží Python interpreter, bez větších rozdílů. Pro správu virtuálního prostředí, kde
jsou knihovny nainstalovány je použit software miniconda. Projekt má strukturu dle standartů
a je psaný objektově orientovaným způsobem.

Aktuálně program skoro umí přijímat data ze zařízení a ukládat je jako trénovací pro
budoucí neuronovou síť. Aktuální stav má několik problémů, jimiž se zabývá spolupracovník
Lukáš Karásek.

\newpage
\printbibliography
\tableofcontents{}
\end{document}
