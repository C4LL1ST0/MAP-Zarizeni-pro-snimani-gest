\documentclass[12pt, a4paper]{article}
\usepackage[utf8]{inputenc}

\usepackage[czech]{babel}
\usepackage[hyphens]{url}
\usepackage[colorlinks=true, urlcolor=blue, linkcolor=black, breaklinks, citecolor=black]{hyperref}
\usepackage{graphicx}

\graphicspath{ {.//obrazky/} }
\usepackage[backend=biber]{biblatex}
\addbibresource{citace.bib}



\sloppy

\begin{document}

\begin{titlepage}
\begin{center}
\thispagestyle{plain}
\pagenumbering{gobble}
\vspace*{1.0cm}

\textbf{\LARGE{Výstup 2 maturitní práce}}
\vspace{2cm}
\hrule
\vspace{0.5cm}
{\Huge \textbf{Zařízení na snímání gest}}
\vspace{0.5cm}
\hrule
\vspace{3cm}

\begin{minipage}{0.4\textwidth}
\begin{flushleft}
  \large{\emph{autor:}\\ Štěpán Bílek}
\end{flushleft}
\end{minipage}
\begin{minipage}{0.4\textwidth}
\begin{flushright}
  \large{\emph{konzultant:} \\ Jaroslav Kořínek}
\end{flushright}
\end{minipage}
\vfill
\large{Školní rok 2025/2026}
\end{center}
\end{titlepage}


\pagenumbering{Roman}
\setcounter{page}{2}

\section{Úvod}
\paragraph{}
Primárním cílem výstupu 2 maturitní práce bylo seznámit se s principem neuronových sítí a navrhnout jednu pro
ovládací program za účelem rozpoznání jednotlivých gest. Cílem sekundárním bylo začít na ovládacím programu
samotném, aby byl schopen přijímat data ze zařízení a ukládat je jako trénovací soubory pro neuronovou síť.

\section{Neuronová síť}
\subsection{Poznatky}
\paragraph{}
Umělá neuronová síť je struktura určená pro distribuované paralelní zpracování dat. Skládá se z umělých neuronů,
jejichž volným předobrazem je biologický neuron. Neurony jsou vzájemně propojeny synaptickými vazbami a navzájem
si předávají signály a transformují je pomocí aktivačních funkcí. \cite{prehled-NN}

Neurony jsou uspořádávány do vrstev. Existuje vrstva vstupní, skrytá (kterých může být více) a výstupní.

\subsubsection{Neurony a synapse}
\paragraph{}
Neuron je funkcí, která má libovolný počet vstupů. Tato funkce je ve své podstatě jednoduchá, přijímané vstupy se
nejdříve vynásobí vahou vázanou k synapsy, ze které přišly, dále se tyto hodnoty sečtou a nakonec je přičten tzv. \uv{bias}.

Synapse je propojení mezi neurony. Ke každé jednotlivé synapsy, kterou neuron má, je přiřazena váha, která začíná
na náhodném čísle a v procesu učení je upřesnǒvána.

Za předpokladu neuronu s 1 vstupem[x], vahou na synapsy[w], bias[b] a jedním výstupem[y] vypadá jeho oprace po matematické stránce takto:
\begin{center}
  \begin{equation}
    y = x*w + b
  \end{equation}
\end{center}
čili se jedná o přímku.

\subsubsection{Aktivační funkce}
\paragraph{}
Je další funkcí v pořadí za neuronem. Tato funkce není nezbytná pro neuronovou síť, ale zpravidla dělá
výstup nelineárním, díky čemuž se síť může naučit komplexnější vzory. Nejčastěji používanými jsou: \textbf{sigmoid},
\textbf{tanh}, \textbf{ReLU} a tomu podobné, \textbf{softmax}. Pro určíté úkoly na různých vrstvách sítě
jsou některé funkce lepší, než jiné. Odvíjí se to od faktorů jako příkrost vzrůstu a H\textsubscript{f}.\cite{af}



\subsubsection{Ztrátová funkce}

\subsubsection{Optimalizační funkce}

\section{Ovládací program}




\newpage
\tableofcontents{}
\printbibliography
\end{document}
